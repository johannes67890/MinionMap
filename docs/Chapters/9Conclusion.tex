\section{Conclusion}
In the beginning of the project we set out to fulfill the given formal requirements of which we interpreted into the requirements found in \textbf{1.3} found section \ref{Introduction/Requirements}. Many of which we succeeded in fulfilling - but due to one of our biggest projects, chunking, not making it into the final version, all tags are stored in the heap, which prevents us from loading Denmark. Additionally, there was not enough time to implement auto correction with the Levenshtein algorithm for address-autocorrection and polygon simplification with the Douglas Peucker algorithm.
\par Despite this, many aspects of the project went well. Having user-based requirements made fulfillment a debate on whether we succeeded or not. As for the system-requirements we can conclude the following.
\par The program runs the entirety of Bornholm smoothly thanks to the K3DTree sorting everything, both in latitude, longitude and hierarchies. The map is being drawn with a simple aesthetic made possible by our Type enumeration of TagWays and TagRelations. Any address can be searched upon, there is even autocompletion as the user types an address. Two addresses can be linked through pathfinding, giving the user a general idea of which roads to choose, when driving, cycling or walking. The map has three different styles, ranging from default, darkmode and grayscale.

In terms of what could be improved upon, there is the implementation of chunking. Which ensures that only a limited amount of tags have to be stored in the heap at once. 

Implementing Douglas Peucker and Levenshtein, would also be a qualified extension for the program.

Additionally, having more coverage of tests on classes such as DrawingMap and the main application, would be preferred to improve upon.

Regarding memory and benchmarks, it would make sense to possibly reduce the loading time of the map, as it already took 4 seconds loading Bornholm. Loading the entirety of Denmark would take minutes, since it’s more than 50 times the size of Bornholm, in terms of Megabytes.

\section{Process Reflection}
As we now have created our map of Denmark, we do have some improvements to our workflow. Through the project, we spend a lot of time on researching different solutions to our problems, which is great when searching for the best solution. But our ability to recreate a certain algorithm within our project is deemed to be difficult, because of the personalized code written earlier. Where implementing and trying out multiple solutions to a problem creates a lot of time wasted on debugging why a specific implementation did not work. If we wanted to recreate the project, we would have made more concise time-frames for how much time we could spend on researching a specific problem, where if we deem to have used too much time, we would have just taken the best solution from the concurrent knowledge. 
Another problem that arose from this is that we would spend a lot of time on a specific requirement while neglecting other equally important requirements, until the final week. Instead, it would make sense to make more lazy solutions, that would be quick and allow us to finish all requirements earlier on in the project, following a long process of optimizing and refactoring.
What we did rather well, was working at a consistent amount of hours throughout the process, of course we spent a little extra in the final week, but overall there were no weeks where we were not productive. We were also great at helping each other, cooperating on making algorithms, which would ensure that each individual would have a broader understanding of the entire system.